
Once \MACI has been compiled and installed it can be run like most other \MPI-based applications:
\begin{lstlisting}[style=SHELL]
mpiexec -np 32 -hostfile $PBS_NODEFILE <PATH TO MACI>/bin/maci [args] \
                              <PYTHON INPUT> <XML SPEC FILE>
\end{lstlisting}
The details of the \MPI start up mechanism will depend on the system. Note that the \XML spec file is only necessary if the the \PYTHON script relies on the high-level \PYTHON library. This is the recommended usage of \MACI. 

Currently, the only argument \MACI takes is \lstinline[style=SHELL]|--input_path "<PATH A>:<PATH B>"|, which defines a list of files where \MACI searches for input (\XML files, \PYTHON files, etc.). This allows to distribute the input data over different directories, which can be helpful sometimes. The option parser is based on the \lstinline[style=CODE]|optparse| module (which is now deprecated it will need to be replaced in the future).

The \lstinline[style=SHELL]|maci| program will take care of the initialization of environment variables (such as \lstinline[style=SHELL]|LD_LIBRARY_PATH| and the \lstinline[style=SHELL]|PYTHONPATH|) for all components of the installation, except for external packages (like \TRILINOS or \PETSC). It is therefore \textbf{not} possible to relocate the \MACI tree after installation. Since a clean build of \MACI requires only a few minutes we believe that the increased comfort of this approach (since there is no need to set \lstinline[style=SHELL]|LD_LIBRARY_PATH| or \lstinline[style=SHELL]|PYTHONPATH| manually) justifies this loss of flexibility.
